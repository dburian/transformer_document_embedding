\chapter{Document representation}\label{chapter:document_representation}

Text embeddings are ubiquitous in Natural Language Processing. However,
embedding long, continuous pieces of text such as documents is substantially
more complex than embedding words or sentences, as the longer and more involved
input still needs to be compressed into a similarly sized vector. In this
chapter, we first briefly explore the use cases of document embeddings.
Afterward, we describe the qualities of document embeddings that we deem
beneficial and build up the motivation behind our training method, which we
describe in Chapter~\ref{chapter:training_method}.

\section{Use cases of document embeddings}

Embeddings that capture the document's semantics in a low-dimensional vector
reduce the noise in the raw input while making the subsequent operations more
efficient. These are the main reasons why document embeddings are widely used
across different tasks such as classification \citep{cohan2020specter,
neelakantan2022text, izacard2021unsupervised, ostendorff2022neighborhood},
ad-hoc search \citep{singh2022scirepeval, zamani2018neural}, query answering
\citep{neelakantan2022text}, visualization \citep{cohan2020specter,
dai2015document} and regression \citep{singh2022scirepeval}.

While some models \citep{singh2022scirepeval} can generate different embeddings
for a single input depending on the task, most embedding models output a single
vector that is effective across many types of tasks \citep{neelakantan2022text,
cohan2020specter, ostendorff2022neighborhood}. This shows that embedding models
can substitute several dedicated models, severely saving time and resources.

\section{Desirable qualities of embeddings}

The usefulness of document embeddings stems from 3 properties. An ideal
document embedding
\begin{enumerate*}[label=(\arabic*)]
  \item represents the document's text faithfully
  \item with a single vector
  \item of low dimension.
\end{enumerate*}

In this section, we focus on faithful document representation, which we view as
a composition of two abstract qualities: \emph{structural} and
\emph{contextual}. Document embedding with structural quality (or structural
document embedding) faithfully models the relationships between words or word
sequences. Based on these relationships, the embedding can capture meaning even
for documents with complex structures. On the other hand, contextual document
embedding composes the meaning of all processed words, capturing the overall
theme or topic of the document. We view these qualities as scales, so a
document embedding may have high structural but low contextual quality.
Such embedding captures the relationships between words very well and thus
faithfully represents the meaning of sections with unambiguous context.
However, the embedding can easily misinterpret sections where context is
needed to disambiguate between several meanings.

Since each document embedding is produced by a model, we may attribute similar
qualities to the models themselves. In this sense, we speak of the model's
\emph{structural} or \emph{contextual capacity}.

In the following subsections, we focus on each quality separately, describing
each in more detail. At the end of this section, we compare the two
qualities and outline our proposed training method described in
Chapter~\ref{chapter:training_method}.

\subsection{Structural quality of document embeddings}

Structural quality defines how well the embedding captures relationships within
the input text. The more complex the relationship is, the higher structural
quality is needed to interpret the text correctly. For instance, we list
exemplary observations based on word relationships in a sentence: ``Fabian
likes playing the guitar, but Rebecca does not.'':

\begin{enumerate}[label={Observation \arabic*.},ref=\arabic*,wide=0pt]

  \item ``Fabian'' likes something based on the words ``Fabian likes''
    \label{obs:fabian_likes}

  \item A guitar can be played based on the words ``playing the guitar''
    \label{obs:guitar_played}

  \item The two sequences of words separated by a comma are in opposition based
    on the words ``, but'' \label{obs:opposed_sequences}

  \item ``Fabian'' likes to play the guitar based on
    Observations~\ref{obs:fabian_likes}~and~\ref{obs:guitar_played}.
    \label{obs:structural_complex}

\end{enumerate}

The relationships get more and more complex as the number of participating
words increases
(Observations~\ref{obs:fabian_likes}-\ref{obs:opposed_sequences}) or as we
layer the relationships (Observation~\ref{obs:structural_complex}).
Therefore, an embedding would need an increasing level of structural quality to
capture Observations~\ref{obs:fabian_likes}-\ref{obs:structural_complex}
correctly.

Embedding can reflect world relationships only if the model that produced it
compares the participating words to each other. Based on the number and
complexity of comparisons the model makes, we can derive its level of
structural capacity. A good example of a model with high structural capacity is
Transformer \citep{vaswani2017attention}. Transformer's self-attention layer
allows each word to exchange information with other words. Additionally,
self-attention allows the aggregation of several words into one. Thanks to
Transformer's layered architecture, such aggregations can be compared similarly
on higher levels. An example of a model with low structural capacity is
Paragraph Vector \citep{le2014distributed}. Paragraph Vector compares words
only in a single fully connected layer. Such architecture prevents the
understanding of more complex relationships that build on other relationships,
such as Observation~\ref{obs:structural_complex}.

\subsection{Contextual quality of document embeddings}


The contextual quality of a document embedding defines how well the embedding
captures the overall meaning of longer texts. The longer the sequence, the
higher the contextual quality of an embedding correctly
capturing its overall topic. For instance, let us consider two
documents: \begin{enumerate*}
  \item a description of a typical commercial turbo-jet airplane
    and\label{enumitem:plane}
  \item a recipe for spicy fried chicken wings\label{enumitem:chicken}
\end{enumerate*}.
A document embedding with high enough contextual quality would reflect that the
following sentence's meaning: ``Left wing is too hot.'' dramatically differs between
the two documents and would accordingly adjust the sentence's contribution to
the resulting document embedding.

Provided the document's text is cohesive and continuous, capturing its overall
meaning gets easier as the text's length increases. Intuitively, the more words
we see, the more information we know about their common theme. As the theme
becomes increasingly more refined, fewer meanings that correspond to it.
Consequently, we judge a model's contextual capacity based on the maximum
length of an input that the model can to process. This number is also commonly
known as the maximum context length of a model. An example of a model with good
contextual capacity is Paragraph Vector \citep{le2014distributed}, which can
process, in theory, indefinitely long sequences\footnote{Provided the
vocabulary size stays constant.}. Additionally, Paragraph Vector stores a
single vector per document, which is iteratively compared to all words within
it. This allows the model to adjust individual words' contribution to the
document's meaning. On the other hand, Transformer \citep{vaswani2017attention}
has much smaller contextual capacity as its memory requirements grow
quadratically with the length of the input, which in practice significantly
shortens Transformer's maximum context length.

\subsection{Combining structural and contextual
qualities}\label{section:combine_structural_and_contextual}

% While each quality describes some aspect of faithful representation, we
% hypothesize that only their composition can achieve true

% Compare the two qualities

Each quality describes a different aspect of faithful representation.
Structural quality is focused more on local relationships of words, while
contextual quality considers mainly the global picture. From a performance
standpoint, structural quality is oriented more toward precision, while
contextual quality is oriented more toward recall. In a way, the two qualities
complement each other. Contextual quality brings in the overall document theme,
while structural quality brings in the detailed meaning of a shorter sequence.
We hypothesize that these two pieces of information can be aligned to produce
precise, unambiguous document embedding that outperforms embeddings
with just a single quality.

While we predict that a mix of both qualities is beneficial, we are unsure
which ratio would be the most performant. Arguably, structural quality is more
important than contextual since, in extreme cases, it can model relationships
so complex, that they span the entire document, substituting contextual
quality’s role. On the other hand, we can expect that, for a given input
length, an embedding model with high structural capacity will be larger than an
embedding model with high contextual capacity. The reason is that the number of
total relationships found in a document grows exponentially with the length of
the document, whereas the number of topics covered can grow only linearly.

Our training method stems from these observations and hypotheses. We align the
two qualities with each other and find the ideal ratio of the two qualities
that produce the best-performing embeddings. We describe our training method
in detail in Chapter~\ref{chapter:training_method}.
